\documentclass[12pt]{article}

\usepackage{sbc-template}
\usepackage{graphicx,url}
\usepackage[brazil]{babel}   
\usepackage[utf8]{inputenc}
\usepackage{amsmath,amsfonts,amssymb,amsthm,epsfig,epstopdf,url,array}
\usepackage[portuguese, ruled,lined,linesnumbered]{algorithm2e}
\usepackage{algorithmic}

\SetKwFor{Para}{para}{fa\c{c}a}{fim para} % Corrigindo um bug do algorithm2e, que escreve os nomes de comandos em espanhol
\SetKwFor{ParaTodo}{para todo}{fa\c{c}a}{fim para todo} % Corrigindo um bug do algorithm2e, que escreve os nomes de comandos em espanhol
\SetKwBlock{Inicio}{início}{fim} % Corrigindo um bug do algorithm2e, que escreve os nomes de comandos em espanhol

\newtheorem{defin}{Definição}
\newenvironment{definicao}[1][]
  {\quote\begin{defin}[#1]}
  {\end{defin}\endquote}

\title{Algoritmos para o 2º trabalho de Linguagens Formais e Compiladores}

\author{Douglas Klein\inst{1}, Lucas Pagotto Tonussi\inst{1}}

\address{Departamento de Informática e Estatística,\\
  Universidade Federal de Santa Catarina (UFSC), Florianópolis, SC, Brasil
  \email{douglas.klein2@gmail.com, lptonussi@gmail.com}
}

\begin{document}

\maketitle

\begin{center}
\begin{tabular}{l r}
Disciplina: & INE5421 - Linguagens Formais e Compiladores\\ Professor: & Olinto
José Varela Furtado
\end{tabular}
\end{center}

\begin{resumo}
Este documento contém algoritmos desenvolvidos no curso da disciplina de
Linguagens Formais e Compiladores, voltados a manipulação de Gramáticas Livres
de Contexto e Autômatos de Pilha, e integrantes do 2º trabalho prático da
disciplina no semestre 2014.2.
\end{resumo}

\section{Liguagem de Programação}
A Linguagem de programação utilizada na implementação dos algoritmos é JavaScript.

\section{Definição da estrutura de dados GLC}
Uma GLC é uma quádrupla, $G=(V_n, V_t, P, S)$, onde:
\begin{enumerate}
	\item $V_n$ é o conjunto de símbolos não-terminais;
	\item $V_t$ é o conjunto de símbolos terminais;
	\item $P$ é o conjunto de produções;
		\begin{itemize}
			\item Cada produção é uma lista, onde:
				\begin{enumerate}
					\item O 1º elemento é uma string que corresponde à cabeça da produção;
					\item O 2º elemento é uma lista de símbolos ($V_n \cup V_t$) que formam o corpo da produção.
				\end{enumerate}
		\end{itemize}
	\item $S \in V_n$ é o símbolo inicial da gramática.
\end{enumerate}


\section{FIRST}
O conjunto $FIRST(X) \subseteq V_t$ contém os símbolos terminais que iniciam uma forma sentencial a partir do símbolo $X$.

\begin{algorithm}[H]
\small
\Entrada{$G=(V_n, V_t, P, S)$ // Uma Gramátic Livre de Contexto }
\Saida{$FIRST$ // Um conjunto de conjuntos FIRST de cada símbolo de G}
\BlankLine
\Inicio{
	$FIRST \gets \emptyset$\;
	\ParaTodo{$t \in V_t$}{
		$FIRST(t) = \{t\}$\;
	}
	\ParaTodo{$X \in V_n$}{
		$FIRST(X) \gets \{a | X \rightarrow a\alpha \wedge a \in V_t \wedge \alpha \in V_n \cup V_t\} \cup \{\epsilon | X \rightarrow \epsilon \in P\}$\;
	}
	$P' \gets \{ X \rightarrow a\alpha | X \rightarrow a\alpha \in P \wedge a \in V_n \wedge \alpha \in V_n \cup V_t\}$\;
	\Repita{$F \neq FIRST$}{
		$F \gets FIRST$\;
		\ParaTodo{$X \rightarrow y_1 y_2 ... y_n \in P'$}{	
			$i \gets 1$\;
			\Repita{$i-1=n \vee \epsilon \notin FIRST(y_{i-1})$}{
				$FIRST(X) \gets FIRST(X) \cup FIRST(y_i)$\;
				$i \gets i + 1$\;
			}
		}
	}
  \Retorna{$FIRST$}\;
}
\caption{\textbf{obterConjuntosFIRST}}
\label{alg:obterConjuntosFIRST}
\end{algorithm}

\end{document}
